\documentclass[a4paper]{scrreprt}

\usepackage[ngerman]{babel}
\usepackage[utf8]{inputenc}
\usepackage[T1]{fontenc}
\usepackage{lmodern}

\usepackage[german,linesnumbered,algoruled,longend,vlined]{algorithm2e}
\DontPrintSemicolon
\SetArgSty{}
\SetKw{KwOr}{or}
\SetKw{KwAnd}{and}
\SetKw{KwNot}{not}
\setlength{\algomargin}{3ex}

\usepackage[fixlanguage]{mybabelbib}
% \selectlanguage{ngerman}
\setbibliographyfont{title}{}
\setbibliographyfont{jtitle}{}
\setbibliographyfont{btitle}{\emph}
\setbibliographyfont{stitle}{\emph}
\setbibliographyfont{journal}{\emph}

\usepackage{amsmath}
\usepackage{amsfonts}
\usepackage{amssymb}
\usepackage{amsthm}

\usepackage{graphicx}
\usepackage[a4paper,bookmarks,bookmarksnumbered]{hyperref}
\usepackage[font=small,format=hang,labelfont=bf,figurename=Abb.,tablename=Tab.]{caption}
\usepackage{enumerate}

\newtheorem{satz}{Satz}[chapter]
\newtheorem{lemma}[satz]{Lemma}
\newtheorem{beobachtung}[satz]{Beobachtung}
\newtheorem{folgerung}[satz]{Folgerung}
\newtheorem{korollar}[satz]{Korollar}
\theoremstyle{definition}
\newtheorem{definition}[satz]{Definition}
\newenvironment{beweis}{\begin{proof}}{\end{proof}}

\graphicspath{{abbildungen/}}

\begin{document}
%%%%%%%%%%%%%%%%%%%%%%%%%%%%%%%%%%%%%%%%%%%%%%%%%%%%%%%%%%%%%%%%%%%%%%%%%%
%%%%%%%%%%%%% Bitte nur ab hier Änderungen vornehmen %%%%%%%%%%%%%%%%%%%%%

%% hier Titel und Autorennamen eintragen

\subject{Bachelorarbeit}
\title{Implementierung von Smooth Orthogonal Drawings planarer Graphen} % Geben Sie hier den Titel Ihrer Arbeit an.
\author{Bernhard Häussner} % Geben Sie Ihren Namen an. 
\date{Eingereicht am XX. YY 20ZZ} % TODO: Geben Sie das Abgabedatum an
\titlehead{Julius-Maximilians-Universität Würzburg\\
Institut für Informatik\\
Lehrstuhl für Informatik I\\
Effiziente Algorithmen und wissensbasierte Systeme}
\publishers{Betreuer:\\
Prof.\ Dr.\ Alexander Wolff\\
Dipl.-Inf.\ Philipp Kindermann} % Geben Sie den Namen des weiteren Betreuers and
\maketitle
\tableofcontents






\chapter{Einleitung}



We want to implement drawing of Smooth Orthogonal Layouts based on an algorithm found in
\cite{smooth-13}. It is based on orthogonal graph drawing of \cite{biedl+kant-98} using an
extension found in \cite{liu+etal-98}.

\chapter{Algorithmus}

\section{High-Level-Pseudocode}

% TODO

\begin{algorithm}[ht]
  \SetKw{True}{true}
  \SetKw{False}{false}
  \caption{SmoothOrthogonalDraw(Graph $G = (V,U)$}
  \label{alg:main}
  \Ein{4-planarer Graph $G = (V,U)$}
  \Aus{$SC_2$-Layout von $G$}
  
  $G_1, \dots, G_n = $BiconnectedComponents$(G)$ \;
  
  \For{$i\leftarrow 1$ \KwTo $n$}{
    $\Gamma_i \leftarrow$ SmoothOrthogonalDrawBiconnected$(G)$ \;
    RightAngelize$(\Gamma_i)$ \; % TODO: better name, doh.
  }
  $\Gamma \leftarrow$ ConnectComponents$(L_1, \dots, L_n)$ \;
  \Return $\Gamma$
\end{algorithm}

\begin{algorithm}[ht]
  \SetKw{True}{true}
  \SetKw{False}{false}
  \caption{SmoothOrthogonalDrawBiconnected(Graph $G = (V,U)$}
  \label{alg:biconnected}
  \Ein{4-planarer, 2-fach knotenzusammenhängender Graph $G = (V,U)$}
  \Aus{$SC_2$-Layout von $G$}
  
  $\Gamma' \leftarrow$ OrthogonalDrawing$(G)$ \;
  EliminateS-Shapes$(\Gamma')$ \;
  $\Gamma \leftarrow$ DrawEdges$(\Gamma')$ \;
  
  \Return $\Gamma$
\end{algorithm}


\chapter{Implementierung}

\chapter{Verbesserungen}

\chapter{Schlussfolgerungen}






\bibliographystyle{mybabalpha-fl}
\bibliography{mybib}

\end{document}
